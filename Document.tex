% Test tex file!

\documentclass[a4paper,12pt]{article}
\usepackage{times}  % DO NOT CHANGE THIS
\usepackage{helvet} % DO NOT CHANGE THIS
\usepackage{courier}  % DO NOT CHANGE THIS
\usepackage[hyphens]{url}  % DO NOT CHANGE THIS
\usepackage{graphicx} % DO NOT CHANGE THIS
\urlstyle{rm} % DO NOT CHANGE THIS
\def\UrlFont{\rm}  % DO NOT CHANGE THIS
\usepackage{natbib}  % DO NOT CHANGE THIS AND DO NOT ADD ANY OPTIONS TO IT
\usepackage{caption} % DO NOT CHANGE THIS AND DO NOT ADD ANY OPTIONS TO IT
\frenchspacing  % DO NOT CHANGE THIS
\setlength{\pdfpagewidth}{8.5in}  % DO NOT CHANGE THIS
\setlength{\pdfpageheight}{11in}  % DO NOT CHANGE THIS
\usepackage{algorithm} %format of the algorithm 
\usepackage{algorithmic} %format of the algorithm 
\usepackage{multirow} %multirow for format of table 
\usepackage{amsmath} 
\usepackage{xcolor}
\usepackage{amssymb}
\usepackage{amsmath}
\usepackage{CJKutf8}
\usepackage{courier}

\begin{document}

\begin{CJK}{UTF8}{gbsn}
% \begin{CJK}{UTF8}{gkai}

\title{强化学习:作业三}

\author{张三 MG20370001}

\date{2020年11月2日}

\maketitle

\section{作业内容}
我们需要在gym Atari环境中实现DQN算法及其变体。本实验的实验环境是Atari Game Pong,Agent需要操控球拍与系统互相击球,未接到球则对方计一分,先取得21分者获胜。实验目标是训练DQN及其变体作为Agent获得游戏胜利,并使获胜时的分差尽可能大。在本次实验中,我分别实现和训练了DQN、Double DQN以及Dueling DQN,并评估了它们在训练过程中的表现和它们在上述游戏中的性能。同时我也实现了Prioritized Replay Buffer,但是完整的算法受限于我的硬件性能无法运行,因此我对论文中的算法进行了一定的简化,并且评估了简化后的算法对DQN训练过程的影响。

\section{实现过程}
\subsection{算法描述}
\paragraph{Q-learning} 在传统Q-learning算法中,我们使用一张 $Q$ 表来记录环境状态 $s$ 以及该状态对应各个动作 $a$ 的长期回报值 $Q(s,a)$,并使用下式来更新 $Q$ 表:
$$
\begin{cases}
	&a'= \mathop{\arg\max}_{x} Q(s',x)\\
	&Q(s,a)=Q(s,a)+\alpha(r+\gamma Q(s',a')-Q(s,a)) 
\end{cases}
$$
其中 $s'$ 是在状态 $s$ 下执行动作 $a$ 后的新状态,$\alpha$ 和 $\gamma$ 分别为学习率和折扣系数。
\paragraph{DQN} 在DQN中,我们不使用表型数据结构记录 $Q$ 值,而是用一个深度神经网络来计算不同的状态-动作对应的 $Q$ 值。DQN相较于传统的Q-learning算法能更好地处理状态-动作空间较大的场景。在DQN中,我们需要最小化TD error,既使网络输出 $Q(s,a)$ 逼近于长期回报的估计值 $r+\gamma Q(s',\mathop{\arg\max}_{x} Q(s',x))$, 其中 $\gamma$ 为折扣系数。我使用均方误差作为损失函数,因此神经网络优化器需要最小化下式:
$$
\|r+\gamma Q(s',\mathop{\arg\max}_{x} Q(s',x))-Q(s,a)\|_2^2
$$
注意到,我们在改变网络参数的时候也会改变优化目标,因此我们需要复制一份原神经网络作为目标网络,从而使优化目标相对稳定。因此改写损失如下:
$$
\|r+\gamma Q'(s',\mathop{\arg\max}_{x} Q'(s',x))-Q(s,a)\|_2^2
$$
其中 $Q$ 为原网络输出, $Q'$ 为目标网络输出。在经过一段时间的训练后,我们需要将原网络参数复制到目标网络上。
\paragraph{Double DQN} 该变体是对DQN中训练目标的优化。
\paragraph{Dueling DQN} 该变体是对DQN中网络结构的优化。
\paragraph{Prioritized Replay Buffer} 优化了Replay Buffer中的采样方式,通过增大TD error较大的样本的采样概率和损失函数权重来提高训练速度。
\subsection{代码实现}
\section{复现方式}
\subsection{训练复现}
复现DQN:在主文件夹下运行 \texttt{python atari\_ddqn.py --train}.
\subsection{测试复现}
\subsection{参数介绍}
\noindent复现其他变体:
\section{实验效果}
\subsection{实验图表展示与分析}
描述累计奖励和样本训练量之间的关系。

\noindent DQN:

\noindent其他变体:

\section{小结}
在这次实验中,我发现...


\section{参考文献}

\end{CJK}
\end{document}
